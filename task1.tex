\section{Исследование графиков}
\begin{sagesilent}
    x = var('x')
    f = log(sqrt(x**2 + 1) / sqrt(x**2 - 1)) / (sin (x))
    log = sqrt(x**2 + 1) / sqrt(x**2 - 1)
\end{sagesilent}

Дана функция $f(x) = \sage{f}$. 

\sageplot{plot(f, xmin = -6.5, xmax = 0, ymin = -10, ymax = 10)}

\subsection{Область определения функции}

Аргумент логарифма должен быть больше 0. $ \sage{log} > 0 $.

$ x \in (-\infty; -1) \cup (1; +\infty)$

Знаменатель $ sin(x) \neq 0$. $\\x \neq \pi k, k \in \mathbb{Z}$

Область определения функции $f(x):D(f(x)) = (-\infty; -1) \cup (1; +\infty) /\ \pi k, k \in \mathbb{Z}$

\subsection{Является ли функция четной или нечетной, является ли периодической}

Функция $f:X \to \mathbb{R}$ называется чётной, если справедливо равенство:\\
$f(-x)=f(x),\quad \forall x\in X$\\
Функция называется нечётной, если справедливо равенство:\\
$f(-x)=-f(x),\quad \forall x\in X$\\
функция называется периодической с периодом  $T\neq 0$, если для каждой точки $x$ из её области определения точки $x+T$ и $x-T$ также принадлежат её области определения, и для них выполняется равенство $f(x)=f(x+T)=f(x-T)$.

Данная функция является нечетной и не является периодической.

\subsection{Точки пересечения графика с осями координат}

Найдем точки пересечения графика функции $f(x)$ с осью абсцисс. Для этого решим уравнение: $\sage{f} = 0$. Корней у данного уравнения нет, функция никогда не пересекает ось абсцисс.

Так как $x \neq 0$ по области определения, следовательно ось ординат функция также не пересекает.

\subsection{Промежутки знакопостоянства}

Рассмотрим интервал $x \in [-2\pi; 0]$

\begin{sagesilent}
	intervals_of_constancy = plot(0, xmin=-6.5, xmax=0, ymin = 0, ymax = 0, axes = False)
	intervals_of_constancy += circle((-2*3.14, 0), 0.05, rgbcolor="black")
	intervals_of_constancy += circle((-3.14, 0), 0.05, rgbcolor="black")
	y_margin = 1
	intervals_of_constancy  += text("+", (-4.5 , y_margin), color="black", fontsize=15) 
	intervals_of_constancy  += text("-", (-1.5 , -y_margin), color="black", fontsize=15)
	
	x = var("x")
	y = log(sqrt(x**2 + 1) / sqrt(x**2 - 1)) / (sin (x))
	d_f = y.diff(x)
	
	plot_interval = plot(d_f, xmin = -3, xmax = 3)
	x1 = 1
	x2 = 2
	x3 = 3
\end{sagesilent}

$df = \sage{d_f}$

\sageplot{plot(intervals_of_constancy)}

\subsection{Промежутки возрастания и убывания}

\sageplot{plot(plot_interval)}

Исследуем функцию на промежутке $[0.5; 1.5]$.
Точки перегиба:

$x_0 = \sage{x1}$

$x_1 = \sage{x2}$

$x_2 = \sage{x3}$

Функция возрастает на промежутках $(0.5; \sage{x1})$ и

$(\sage{x2};\sage{x3})$.

Убывает при $x \in (\sage{x1}; \sage{x2})$.
\newpage
\subsection{Точки экстремума и значения в этих точках}

График производной функции:

\begin{sagesilent}
	var("x")
	y(x) = log(sqrt(x**2 + 1) / sqrt(x**2 - 1)) / (sin (x))
	d_f = y.diff(x)
	graf = plot(d_f, -6.5, 1)
	graf += point((-x1, 0), size = 30, color = 'black')
	graf += point((-x2, 0), size = 30, color = 'black')
	graf += point((-x3, 0), size = 30, color = 'black')
\end{sagesilent}

\sageplot{plot(graf)}

\textbf{Необходимые условия существования экстремума:}

Пусть точка $x_{0}$ является точкой экстремума функции $f$, определенной в некоторой окрестности точки $x_{0}$.
Тогда либо производная $f'(x_{0})$ не существует, либо $f'(x_{0})=0$.

Эти условия не являются достаточными, так, функция может иметь нуль производной в точке, но эта точка может не быть точкой экстремума, а являться, скажем, точкой перегиба.

\textbf{Достаточные условия существования экстремума:}

1. Если функция непрерывна в окрестности точки или  не существует и производная  при переходе через точку  меняет свой знак, тогда в точке  функция  имеет экстремум, причем это минимум, если при переходе через точку  производная меняет свой знак с минуса на плюс; максимум, если при переходе через точку  производная меняет свой знак с плюса на минус.

2. Если функция непрерывна в окрестности точки, $f'(x_{0}) = 0$, а $f''(x_{0}) \neq 0$, то в $x_0$  достигается экстремум, причем, если $f''(x_{0}) > 0$ , то в точке функция имеет минимум; если $f''(x_{0}) < 0$, то в точке  функция  достигает максимум.

\begin{sagesilent}
	var("x")
	y(x) = sin(2*x**3)**2/x**3 + x
	d_f = y.diff(x)
	d_f2 = d_f.diff(x)
\end{sagesilent}
~\\

Исследуем $x_0 = \sage{x1}$. $f''(x_0) = \sage{d_f2(x1).n(digits = 5)}$. Значение меньше нуля --- точка является локальным максимумом.

Далее $x_1 = \sage{x2}$. $f''(x_1) = \sage{d_f2(x2).n(digits = 5)}$. Значение больше нуля --- точка является локальным минимумом.

Наконец, исследуем $x_2 = \sage{x3}$. $f''(x_2) = \sage{d_f2(x3).n(digits = 5)}$. Значение меньше нуля --- точка является локальным максимумом.

\subsection{Непрерывность. Наличие точек разрыва и их классификация}

Если в точке имеются конечные пределы, но они не равны $f(x_0+0) \neq f(x_0-0)$, то $x_0$ называется точкой разрыва первого рода.

Точками разрыва второго рода называются точки, в которых хотя бы один из односторонних пределов равен $\infty$ или не существует.

Данная функция имеет точку разрыва первого рода при $x=0$

\subsection{Асимптоты. Найти необходимые пределы, построить асимптоты на графике}

Вертикальные асимптоты проходят через точки разрыва (когда предел функции в этой точке стремится к бесконечности). 

Горизонтальные асимптоты проходят через точку с координатой $х$, которая является пределом функции при устремлении аргумента функции к бесконечности.

Наклонные асимптоты задаются графиком прямой $kx+b$, где за наклон будет отвечать предел отношения функции к аргументу, где последний стремится к бесконечности. За смещение отвечает предел функции $f(x)-kx$, где $x$ стремится к бесконечности.

У данной функции существуют вертикальные асимптоты.\\$x=\pi k, k \in \mathbb{Z};$\\$x = -1;$\\$x = 1$.
\begin{sagesilent}
	asimpt = plot(f, xmin = -6.5, xmax = 0, ymin = -5, ymax = 5)
	asimpt += parametric_plot((-1, x), (x, -5, 5))
\end{sagesilent}

\sageplot{asimpt}

