\section{Линейный оператор и базисы}

\begin{sagesilent}
def to_x(_z, _a):
    return _z - _a/3
\end{sagesilent}

\begin{sagesilent}
import copy as cp
A = Matrix([[2, 0, -1], [1, 1, -1], [-1, 0, 2]])
B = Matrix([[0], [0], [0]])
R.<x> = QQ[]
phi_A = det(A - x*identity_matrix(3)).monic()
# метод Кардано для поиска собственных чисел
x, a, b, c, d = var("x", "a", "b", "c", "d")
a = 1
b = -5
c = 7
d = -3
y = a*x**3 + b*x**2 + c*x + d;
var("p", "q")
p = (3*a*c - b**2)/(3*a**2)
q = ((2*b**3)/(27*a**3)) - ((b*c)/(3*a**2))+ (d/a)
Q = ((p/3)**3) + ((q/2)**2)
alpha = ((-q/2) + sqrt(Q))**(1/3)
beta = ((-q/2) - sqrt(Q))**(1/3)
epsilon = -1/2 + (sqrt(-3))/2

res = [alpha + beta
, alpha*epsilon + beta*epsilon**2
, beta*epsilon + alpha*epsilon**2
]

result_phi = []
result_matrix = []
vectors = []
result_vectors = []

for i, z_i in enumerate(res):
    x_i = to_x(_z = z_i, _a = b/a)
    result_phi.append(x_i.n(digits=2).real())


for i, lam1 in enumerate(result_phi):
    mat = A - (identity_matrix(3) * lam1)
    mat = mat.echelon_form()
    result_matrix.append(mat)

for i, lam1 in enumerate(result_phi):
    vectors.append(result_matrix[i].right_kernel().basis())
    result_vectors.append(vectors[i][0])

new_basis = Matrix([[1, 1, 0], [1, -1, 1], [0, 1, -1]])
new_basis_result = []
for i, lam1 in enumerate(result_vectors):
    new_basis_result.append(new_basis.inverse() * result_vectors[i])
    
S = Matrix([[1, 1, -1], [1, 0, 1], [0, 1, 0]])
S = S.transpose()
v2 = Matrix(QQ, [0.00, 1.0, 0.00])
A = Matrix([[2,0,-1],[1,1,-1],[-1,0,2]])
J = S.inverse() * A * S
J_sage = A.jordan_form()

M = (J * A.inverse()).inverse()

coefficients_ = phi_A.coefficients()
for i in range(4):
    coefficients_[i] = coefficients_[i] * -1
    
A1 = (coefficients_[3]*A**2 + coefficients_[2]*A + coefficients_[1]*identity_matrix(3))/-coefficients_[0]
A3 = coefficients_[2]*A*A + coefficients_[1]*A + coefficients_[0]


\end{sagesilent}

Линейный оператор A задан в каноническом базисе матрицей.
\begin{equation*}
	A = \sage{A}
\end{equation*}

Найдем собственные числа матрицы, решив методом Кардано характеристическое уравнение:

\begin{center}
	$\phi_A = \sage{phi_A};$
	
    $x_0 = \sage{result_phi[0]};$
    
	$x_1 = \sage{result_phi[1]};$
	
	$x_2 = \sage{result_phi[2]}.$
\end{center}

Приведем получившиеся матрицы к ступенчатому виду, чтобы найти собственные векторы $A$.

\begin{center}
	$A_{\lambda_0} = \sage{result_matrix[0]};$
	отсюда собственный вектор $\sage{result_vectors[0]}$
	
	$A_{\lambda_1} = \sage{result_matrix[1]};$
	отсюда собственный вектор $\sage{result_vectors[1]}$
	
	$A_{\lambda_2} = \sage{result_matrix[2]};$
	отсюда собственный вектор $\sage{v2}$
\end{center}

Найдем найденные собственные векторы в базисе $B = \sage{new_basis}$. Для этого воспользуемся формулой:

\begin{center}
	$V' = B^{-1}  V$
	
	$V'_{\lambda_0} = \sage{new_basis_result[0]}$
	
	$V'_{\lambda_1} = \sage{new_basis_result[1]}$

	$V'_{\lambda_2} = \sage{(new_basis.inverse() * v2.transpose()).transpose()}$
\end{center}

Жордановой матрицей называют блочно-диагональную матрицу, на диагонали которой стоят жордановы клетки.

Составим матрицу из собственных векторов.
\begin{center}
	$S = \sage{S}$
\end{center}

Для перехода к жордановой форме транспонируем $S$ и воспользуемся формулой:
\begin{center}
	$J = S^TAS$
\end{center}

Отсюда жорданова форма матрицы $A$:
\begin{center}
 	$J = \sage{J}$
\end{center} 

Найдем матрицу перехода от матрицы $A$ к жорданову базису 
~\\
воспользовавшись формулой:

\begin{center}
	$M = (BA^{-1})^{-1}$
\end{center}

Исходя из этого матрица перехода к жордановой форме:

\begin{center}
	$M_J = (JA^{-1})^{-1} = \sage{M}$
\end{center}

Вычислим $A^{-1}$ и $A^3$ по теореме Кэли - Гамильтона.

Теорема Кэли - Гамильтона утверждает, что любая квадратная матрица удовлетворяет своему характеристическому уравнению, и как следствие,
~\\
 обуславливает существование аннулирующего многочлена. Выпишем
~\\ 
характеристическое 
уравнение:
\begin{center}
	$\phi_A = \sage{phi_A}$, откуда следует, что
	
	$A^3 = 5A^2 - 7A + 3E$
\end{center}

Вычисляя $A^2$ как $A*A$ и суммируя с остальными слагаемыми, получаем:
\begin{center}
	$A^3 = \sage{5*A*A} - \sage{7*A} + \sage{3*identity_matrix(3)}$;
	
	$A^3 = \sage{5*A*A - 7*A + 3*identity_matrix(3)}$
\end{center}
~\\
~\\

Чтобы вычислить $A^{-1}$ домножим слева все уравнение на  $A^{-1}$:

\begin{center}
		$A^3 - 5A^2 + 7A - 3E = 0 $ $\vert$ $\cdot A^{-1}$
		~\\
		$A^2 - 5A + 7E - 3A^{-1} = 0$
		~\\
		$A^{-1} = (A^2 - 5A + 7E) / 3$
		~\\
		~\\
		$A^{-1} = (\sage{A*A} - \sage{5*A} + \sage{7*identity_matrix(3)}) / 3$
		~\\
		~\\
		$A^{-1} = \sage{(A*A - 5*A + 7*identity_matrix(3)) / 3}$
\end{center}

