\section{Решение алгебраических уравнений четвертой степени}
\begin{sagesilent}

	x = var("x")
	f_x = x**4 - 4*x**3 - 5*x**2 + 7*x - 3
	a = -4
	b = -5
	c = 7
	d = -3
	y = var("y")
	polynom = f_x(x = y - 3/4).expand().simplify_full()
	
	pqr = {'p':-59/8, 'q': 51/8, 'r': 525/256}
	var("s p q r")
	resolvent = 2*s**3 - p*s**2 - 2*r*s + r*p - q**2/4

	poly_s_n = resolvent(**pqr)
	
	sols = solve(poly_s_n, s)  
	s_0 = sols[2].rhs()
	
	var("y s p q")
	poly_y_1 = y**2 - y*sqrt(2*s - p) + q/(2*sqrt(2*s - p)) + s
	poly_y_2 = y**2 + y*sqrt(2*s - p) - q/(2*sqrt(2*s - p)) + s
	
	poly_y_1_n = poly_y_1(**pqr, s=s_0)
	poly_y_2_n = poly_y_2(**pqr, s=s_0)
	
	sols = solve(poly_y_1_n, y)
	sols.extend(solve(poly_y_2_n, y))
	
	sols = solve(f_x, x)
	graf = plot(f_x, -5, 5)
	complex_graf = complex_plot(f_x, (-5, 5), (-5, 5))
	
	answers = []
	
	for i, sol in enumerate(sols):
	    a = sol.rhs()
	    if (abs(a.imag()) < 0.00001):
	    	answers.append(a.real())
	    else:
	        answers.append(a)

	for i in answers:   
	    if (i.imag() == 0):
	    	graf += point((i, 0), size = 40, color = 'black')
	
\end{sagesilent}
Дано уравнение: $\sage{f_x} = 0$

Метод Феррари состоит из двух этапов.

На первом этапе уравнения вида 
\begin{center}
	$a_0x^4 + a_1x^3 + a_2x^2 + a_3x + a_4 = 0$,
    
    где $a_0, a_1, a_2, a_3, a_4$ - произвольные числа, причем $a_0 \neq 0$.
\end{center} 

приводятся к уравнениям четвертой степени, у которых отсутствует член с третьей степенью неизвестного.

На втором этапе полученные уравнения решаются при помощи разложения на множители, однако для того, чтобы найти требуемое разложение на множители, приходится решать кубические уравнения, например методом Кардано.

Пусть уравнение имеет вид $x^4+ax^3 + bx^2 + cx + d = 0$:

Произведем замену:
$y = x -\frac{a}{4}$, где $a$ --- коэффициент при переменной в третьей степени уравнения с коэффициентом при старшей степени равным 1.

$ y = x - 3/4 $, тогда заданное уравнение имеет вид: 

$\sage{polynom} = 0$.

Найдем значение коэффициентов $p, q, r$ для исходного уравнения, чтобы уравнение приняло вид: $y^4 + py^3 + qy + r = 0$:

$p = b- \frac{3a^2}{8} = \sage{(b - (3*a**2)/8).n(digits = 5)}$;

$q = \frac{a^3}{8} - \frac{ab}{2} + c = \sage{((a**3) / 8 - (a*b)/2 + c).n(digits = 5)}$;

$r = \frac{-3a^4}{256} + \frac{a^2b}{16} - \frac{ca}{4} + d = \sage{((-3*a**4 )/ 256 +( a**2*b) / 16 - (c*a) / 4 + d).n(digits = 5)}$.

Добавив и вычитая в левой части уравнения выражение $2sy^2 + s^2$ получим:
$\sage{resolvent}$

Это выражение называется кубическая резольвента. Найдем хотя бы одно значение $s$ из уравнения $\sage{poly_s_n} = 0$.

$s_0 = \sage{s_0}$

Получаем два уравнения из которого вычисляется $y$, а затем с помощью обратной замены --- $x$.

$\sage{poly_y_1_n} = 0$ ; $\sage{poly_y_2_n} = 0$
~\\

$x_0 = \sage{answers[0].n(digits = 5)}$

$x_1 = \sage{answers[1].n(digits = 5)}$

$x_2 = \sage{answers[2].n(digits = 5)}$

$x_3 = \sage{answers[3].n(digits = 5)}$

\sageplot{graf}

\sageplot{complex_graf}