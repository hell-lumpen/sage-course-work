\section{СЛАУ}
\begin{equation*}
	\begin{cases}
		2x_1 - x_2 - x_3 = 4, 
		\\
		3x_1 + 4x_2 - 2x_3 = 11,
		\\
		3x_1 - 2x_2 + 4x_3 = 11, 
	\end{cases}
\end{equation*}

\subsection{Метод Крамера}

\begin{sagesilent}
	matrix_a = matrix([[2,-1,-1],[3,4,-2],[3,-2,4]])
	matrix_b = matrix([[4],[11],[11]])
\end{sagesilent}

Решим данную систему методом Крамера.
Получаем матрицу коэффициентов
$A = \sage{matrix_a}$ и столбец свободных членов $B = \sage{matrix_b}$.
$D = det(A) = \begin{vmatrix}
	2 & -1 & -1\\
	3 & 4 & -2\\
	3 & -2 & 4\\
\end{vmatrix} = \sage{matrix_a.det()}$.

$D \neq 0$, следовательно система уравнений имеет решение. В определителях столбец коэффициентов при соответствующей неизвестной заменяется столбцом свободных членов системы.

$D_1 = \begin{vmatrix}
	4 & -1 & -1\\
	11 & 4 & -2\\
	11 & -2 & 4\\
\end{vmatrix} = 180$, 
$D_2 = \begin{vmatrix}
	2 & 4 & -1\\
	3 & 11 & -2\\
	3 & 11 & 4\\
\end{vmatrix} = 60$,
$D_3 = \begin{vmatrix}
	2 & -1 & 4\\
	3 & 4 & 11\\
	3 & -2 & 11\\
\end{vmatrix} = 60$.

Найдем $x_1 = D_1/D = 3$, $x_2 = D_2/D = 1$, $x_3 = D_3/D = 1$.
\subsection{Метод Гаусса}

Для решения СЛАУ методом Гаусса сначала необходимо проверить совместность системы по теореме Кронекера-Капелли. Для того чтобы линейная система являлась совместной, необходимо и достаточно, чтобы ранг расширенной матрицы этой системы был равен рангу её основной матрицы.

Система уравнений разрешима тогда и только тогда, когда $ \operatorname {rank} A=\operatorname {rank} (A|B)$, где $(A|B)$ — расширенная матрица, полученная из матрицы $A$ приписыванием столбца $B$.

\begin{sagesilent}
	matrix_a = matrix([[2,-1,-1],[3,4,-2],[3,-2,4]])
	matrix_b = matrix([[4],[11],[11]])
	
	new_mat = block_matrix([matrix_a, matrix_b], nrows=1)

	a_rank = matrix_a.rank()
	new_rank = new_mat.rank()
	new_mat = new_mat.echelon_form()
\end{sagesilent}
Ранг матрицы коэффициентов $ \operatorname {rank} A= \sage{matrix_a.rank()}$.

Ранг расширенной матрицы $ \operatorname {rank} (A|B) = \sage{new_mat.rank()}$.

Ранги матриц равны следовательно система совместна.
\begin{sagesilent}
	line = 3
	column  = line + 1 
	
	x_3 = new_mat[line-1][column-1]/new_mat[line-1][column-2]
	x_2 = (new_mat[line-2][column-1] - x_3*new_mat[line-2][column-2])/new_mat[line-2][column-3]
	x_1 = (new_mat[line-3][column-1]- x_3*new_mat[line-3][column-2] - x_2*new_mat[line-3][column-3])/new_mat[line-3][column-4]
	
	ans = matrix([[x_1],[x_2],[x_3]])
\end{sagesilent}

Далее необходимо привести матрицу к диагональному виду 
$\sage{block_matrix([matrix_a, matrix_b], nrows=1)} \sim \begin{pmatrix}
	3 & 4 & -2& \vline & 11\\
	0 & -6 & 6 & \vline & 0\\
	0 & 0 & -10/3 & \vline & -10/3\\
\end{pmatrix} \sim \sage{ans}$.
