\section{Численные методы — Интегралы}

\begin{sagesilent}
	import numpy
	var("x")
	
	y = (8*x - arctan(2*x)) / (4*x**2 + 1)
	a = 0 # нижний предел
	b = e # верхний предел
	
	plot1 = plot(y, a, b)
	
	fill_plot1 = plot(y, a, b, fill = True, fillcolor = "red", title = "Необходимо найти площадь закрашенной фигуры")
	
	# метод прямоугольников
	
	dx = 0.1 # приращение x
	
	graph_rectangle = plot(y, a, b)
	result_rectangle = 0
	rectangles = []
	iteration_rect = []
	
	for i, xi in enumerate(numpy.arange(a, b , dx)):
	    dy = y(xi) # приращение y
	    result_rectangle += dx*dy; # площадь маленького прямоугольника из которых будет состоять общая площадь
	
	    text_ = text(r"$i={}, curr={}, result={}$".format(i, str((dx*dy).n(digits=4)),str(result_rectangle.n(digits=4))), (1,-0.5), fontsize=12, color="black")
	
	    rectangles.append(
	        polygon2d([(xi, 0),
	        (xi + dx, 0),
	        (xi + dx, dy),
	        (xi, dy)])
	        )
	    graph_rectangle += plot(rectangles[-1])

	    if (i < 5):
	        iteration_rect.append(graph_rectangle)
	    
	
# метод трапеций

dx_ = 0.3

graph_trapezoid = plot(y, a, b)
result_trapezoid = 0
trapezoids = []
iteration_trap = []


for i, xi in enumerate(numpy.arange(a, b , dx_)):
    dy1 = y(xi)
    dy2 = y(xi + dx_)
    result_trapezoid += (dy1 + dy2) / 2 * dx_ # площадь маленькой трапеции из которых будет состоять общая площадь

    trapezoids.append(
        polygon2d([(xi, 0),
        (xi + dx_, 0),
        (xi + dx_, dy2),
        (xi, dy1)])
    )

    graph_trapezoid += plot(trapezoids[-1])

    if i < 5:
        iteration_trap.append(graph_trapezoid)

result_sage, e = numerical_integral(y, a, b)

\end{sagesilent}

\begin{minipage}{0.4\textwidth}
	Найти площадь закрашенной фигуры методом трапеций и прямоугольников. Площадь численно равна интегралу:
	$\int\limits_{\sage{a}}^{\sage{b}} \sage{y} dx$.
\end{minipage}
\hfill
\begin{minipage}{0.4\textwidth}
	\sageplot[width=7cm]{plot(fill_plot1)}
\end{minipage}

\subsection{Метод прямоугольников}

Отрезок интегрирования разбивается на равные части отрезки длины: $\Delta x = \frac{b−a}{n}$. Данная величина называется шагом разбиения. С помощью шага разбиения можно регулировать точность вычисления интеграла. В результате получим точки:
$x_0 = a < x_1 < x_2 < \dots < x_{n−1} = b$.
Тогда значение интеграла будет суммой площадей всех маленьких прямоугольников:

\begin{center}
	$\sum\limits_{i=1}^n  f(x_n)\cdot \Delta x_n$

\sageplot[width=7cm]{plot(iteration_rect[0])}
\sageplot[width=7cm]{plot(iteration_rect[1])}

\sageplot[width=7cm]{plot(iteration_rect[2])}
\sageplot[width=7cm]{plot(iteration_rect[3])}
\end{center}

Значение интеграла полученное методом прямоугольников с шагом разбиения $\sage{dx.n(digits = 2)}$.

\begin{center}
	$\int\limits_{\sage{a}}^{\sage{b}} \sage{y} dx = \sage{result_rectangle}$.
\end{center}

\subsection{Метод трапеций}

Отрезок интегрирования разбивается на равные части отрезки длины: $\Delta x = \frac{b−a}{n}$.

Численно интеграл будет равен сумме площадей маленьких прямоугольных трапеций:

\begin{center}
	$\sum\limits_{i=1}^n  \frac{f(x_n) + f(x_n + \Delta x_n)}{2} \cdot \Delta x_n$
	
	\sageplot[width=7cm]{plot(iteration_trap[0])}
	\sageplot[width=7cm]{plot(iteration_trap[1])}
	
	\sageplot[width=7cm]{plot(iteration_trap[2])}
	\sageplot[width=7cm]{plot(iteration_trap[3])}
\end{center}

Значение интеграла полученное методом трапеций с шагом разбиения $\sage{dx_.n(digits = 2)}$.

\begin{center}
	$\int\limits_{\sage{a}}^{\sage{b}} \sage{y} dx = \sage{result_trapezoid}$.
\end{center}

Результат, полученный встроенными средствами Sage:
\begin{center}
	$\int\limits_{\sage{a}}^{\sage{b}} \sage{y} dx = \sage{result_sage}$.
\end{center}